\documentclass[12pt]{article}
\usepackage{amsmath, amsthm, amsfonts, amssymb}
\usepackage{mathtools} % For dcases and \xrightarrow
\usepackage[utf8]{inputenc}
\usepackage[all,cmtip]{xy}
\usepackage{xcolor}
\usepackage{graphicx}
\graphicspath{ {./images/} }
\usepackage{url}
\usepackage{csquotes}

% WEIRD: bibtex didn't like the cite key ``humphreys-BGG''
% But was OK when I changed it to ``humphreys.BGG'' for like a minute.
% Then it stopped working again.

%\usepackage{babel}[portuguese]
%\usepackage{geometry}
%\usepackage{cancel}

\usepackage{tikz}
%\usepackage{bbm}


\usetikzlibrary{decorations.markings}

\newcommand\Ccancel[2][black]{\renewcommand\CancelColor{\color{#1}}\cancel{#2}}
\newcommand{\vvec}[1]{\vert #1 \rangle}

%%%%%%%%%%%%%%%%%%%%%%%%%%%%%%%%%%%%%%%%%%%%%%%%%%%%%%%%%%%%%%%%%%%%%%%%%%%%%%%%%%%%%%%%
%%%%%%%%%%%%%%%%%%%%%%%%%%%%%%%%%%%%%%%%%%%%%%%%%%%%%%%%%%%%%%%%%%%%%%%%%%%%%%%%%%%%%%%%

\theoremstyle{plain}
\newtheorem{thm}{Theorem}[section]
\newtheorem{lemma}[thm]{Lemma}
\newtheorem{prop}[thm]{Proposition}
\newtheorem{cor}[thm]{Corollary}
\newtheorem{rmrk}[thm]{Remark}
\newtheorem{rem}[thm]{Remark}
\newtheorem{exer}[thm]{Exercise}


\theoremstyle{definition}
\newtheorem{defn}{Definition}[section]
\newtheorem{exmp}{Example}%[section]



\numberwithin{equation}{section}



% LECTURE TAG
\newcommand{\lectag}[2]{\textbf{\color{blue}{#1}. Lecture #2}}

% THE SQUARE BOX AT THE TOP OF THE PAGE
\newcommand{\squarething}[1]{
  \noindent
  \begin{center}
  \framebox{
    \vbox{
      \vspace{4mm}
      \hbox to 5.78in { {\Large \hfill #1  \hfill} }
      \vspace{4mm}
    }
  }
  \end{center}
  \vspace*{4mm}
}

% 1-inch margins, from fullpage.sty by H.Partl, Version 2, Dec. 15, 1988.
\topmargin 0pt
\advance \topmargin by -\headheight
\advance \topmargin by -\headsep
\textheight 8.9in
\oddsidemargin 0pt
\evensidemargin \oddsidemargin
\marginparwidth 0.5in
\textwidth 6.5in

\parindent 0in
\parskip 1.5ex

%%%%%%%%%%%%%%%%%%%%%%%%%%%%%%%%%%%%%%%%%%%%%%%%%%%%%%%%%%%%%%%%%%%%%%%%%%%%%%%%%%%%%%%%
%%%%%%%%%%%%%%%%%%%%%%%%%%%%%%%%%%%%%%%%%%%%%%%%%%%%%%%%%%%%%%%%%%%%%%%%%%%%%%%%%%%%%%%%



\DeclareMathOperator{\Gode}{Gode}
\DeclareMathOperator{\coker}{coker}
\DeclareMathOperator{\spec}{Spec}
\DeclareMathOperator{\sheaves}{Sh}
\DeclareMathOperator{\presheaves}{PreSh}
\DeclareMathOperator{\Ext}{Ext}
\DeclareMathOperator{\Der}{Der}
\DeclareMathOperator{\Stab}{Stab}



\DeclareMathOperator{\shom}{\underline{\Hom}}
\DeclareMathOperator{\Sym}{Sym}

\DeclareMathOperator{\Ad}{Ad}
\DeclareMathOperator{\ad}{ad}
\DeclareMathOperator{\supp}{supp}
\DeclareMathOperator{\Tor}{Tor}
\DeclareMathOperator{\Mor}{Mor}
\DeclareMathOperator*{\colim}{colim}
\DeclareMathOperator{\ch}{Ch}
\DeclareMathOperator{\en}{End}
\DeclareMathOperator{\rk}{Rank}
\DeclareMathOperator{\mult}{mult}
\DeclareMathOperator{\height}{ht}
\DeclareMathOperator{\trans}{{\sf T}}
\DeclareMathOperator{\tr}{Tr}
\DeclareMathOperator{\Hom}{Hom}
\DeclareMathOperator{\sheafHom}{\underline{Hom}}
\DeclareMathOperator{\gr}{gr}
\DeclareMathOperator{\res}{Res}
\DeclareMathOperator{\zhu}{Zhu}
\DeclareMathOperator{\lie}{Lie}
\DeclareMathOperator{\aut}{Aut}
\DeclareMathOperator{\Ind}{Ind}
\DeclareMathOperator{\Id}{Id}
\DeclareMathOperator{\wt}{wt}
\DeclareMathOperator{\img}{im}
\DeclareMathOperator{\Mat}{Mat}
\DeclareMathOperator{\Et}{\acute{E}t}

\DeclareMathOperator{\ev}{ev}
\DeclareMathOperator{\coev}{coev}

\DeclareMathOperator{\BI}{BI}
\DeclareMathOperator{\NI}{NI}
\DeclareMathOperator{\LNI}{LNI}
\DeclareMathOperator{\RNI}{RNI}





\newcommand{\V}{\mathcal{V}}

\newcommand{\what}[1]{\widehat{#1}}
\newcommand{\wtil}[1]{\widetilde{#1}}
\newcommand{\ov}[1]{\overline#1}

\newcommand{\fu}{\mathfrak{u}}
\newcommand{\op}{op}
\newcommand{\M}{\mathcal{M}}
\newcommand{\bbD}{\mathbb{D}}
\newcommand{\im}{im}

\newcommand{\m}{\mathfrak{m}}

\newcommand{\bbA}{\mathbb{A}}


\newcommand{\al}{\alpha}
\newcommand{\la}{\lambda}
\newcommand{\La}{\Lambda}
\newcommand{\Om}{\Omega}
\newcommand{\eps}{\epsilon}
\newcommand{\D}{\Delta}

\newcommand{\C}{\mathbb{C}}
\newcommand{\F}{\mathbb{F}}
\newcommand{\Q}{\mathbb{Q}}
\newcommand{\R}{\mathbb{R}}
\newcommand{\Z}{\mathbb{Z}}
\newcommand{\bbP}{\mathbb{P}}


\newcommand{\frakb}{\mathfrak{b}}
\newcommand{\g}{\mathfrak{g}}
\newcommand{\h}{\mathfrak{h}}
\newcommand{\n}{\mathfrak{n}}
\newcommand{\gll}{\mathfrak{gl}}
\newcommand{\sll}{\mathfrak{sl}}
\newcommand{\spp}{\mathfrak{sp}}
\newcommand{\soo}{\mathfrak{so}}

\newcommand{\CA}{\mathcal{A}}
\newcommand{\CC}{\mathcal{C}}
\newcommand{\CD}{\mathcal{D}}
\newcommand{\CE}{\mathcal{E}}
\newcommand{\CF}{\mathcal{F}}
\newcommand{\CG}{\mathcal{G}}
\newcommand{\CL}{\mathcal{L}}
\newcommand{\CP}{\mathcal{P}}
\newcommand{\CR}{\mathcal{R}}
\newcommand{\CV}{\mathcal{V}}
\newcommand{\CW}{\mathcal{W}}
\newcommand{\OO}{\mathcal{O}}
\newcommand{\CU}{\mathcal{U}}
\newcommand{\CM}{\mathcal{M}}

\newcommand{\Ass}{\mathbf{Ass}}
\newcommand{\Lie}{\mathbf{Lie}}
\newcommand{\Set}{\mathbf{Set}}
\newcommand{\Vect}{\mathbf{Vec}}



\newcommand{\ma}{\mathfrak{a}}

\newcommand{\Vir}{Vir}

\newcommand{\vac}{\left|0\right>}

\newcommand{\ket}[1]{\left|#1\right>}

\newcommand{\stacksubscript}[2]{\genfrac{}{}{0pt}{}{#1}{#2}}

\newcommand{\tbt}[4]{\left(\begin{array}{cc} #1 & #2 \\ #3 & #4 \\ \end{array}\right)}

%%%%%%%%%%%%%%%%%%%%%%%%%%%%%%%%%%%%%%%%%%%%%%%%%%%%%%%%%%%%%%%%%%%%%%%%%%%%%%%%%%%%%%%%
%%%%%%%%%%%%%%%%%%%%%%%%%%%%%%%%%%%%%%%%%%%%%%%%%%%%%%%%%%%%%%%%%%%%%%%%%%%%%%%%%%%%%%%%
%%%%%%%%%%%%%%%%%%%%%%%%%%%%%%%%%%%%%%%%%%%%%%%%%%%%%%%%%%%%%%%%%%%%%%%%%%%%%%%%%%%%%%%%
%%%%%%%%%%%%%%%%%%%%%%%%%%%%%%%%%%%%%%%%%%%%%%%%%%%%%%%%%%%%%%%%%%%%%%%%%%%%%%%%%%%%%%%%
%%%%%%%%%%%%%%%%%%%%%%%%%%%%%%%%%%%%%%%%%%%%%%%%%%%%%%%%%%%%%%%%%%%%%%%%%%%%%%%%%%%%%%%%
%%%%%%%%%%%%%%%%%%%%%%%%%%%%%%%%%%%%%%%%%%%%%%%%%%%%%%%%%%%%%%%%%%%%%%%%%%%%%%%%%%%%%%%%

\begin{document}

\squarething{Introduction to Lie algebras}

Based on V. Kac's classic course at MIT.

Jethro van Ekeren \\


%%%%%%%%%%%%%%%%%%%%%%%%%%%%%%%%%%%%%%%%%%%%%%%%%%%%%%%%%%%%%%%%%%%%%%%%%%%%%%%%
%%%%%%%%%%%%%%%%%%%%%%%%%%%%%%%%%%%%%%%%%%%%%%%%%%%%%%%%%%%%%%%%%%%%%%%%%%%%%%%%
%%%%%%%%%%%%%%%%%%%%%%%%%%%%%%%%%%%%%%%%%%%%%%%%%%%%%%%%%%%%%%%%%%%%%%%%%%%%%%%%
%%%%%%%%%%%%%%%%%%%%%%%%%%%%%%%%%%%%%%%%%%%%%%%%%%%%%%%%%%%%%%%%%%%%%%%%%%%%%%%%
%%%%%%%%%%%%%%%%%%%%%%%%%%%%%%%%%%%%%%%%%%%%%%%%%%%%%%%%%%%%%%%%%%%%%%%%%%%%%%%%

%\lectag{2023-01-09}{01}


%\tableofcontents

\section{Preliminaries on linear algebra}



\section{Lie algebras}

\begin{defn}
An algebra over a field $\F$ consists of a vector space $A$ over $\F$ and a function $* : A \times A \rightarrow A$ which is $\F$-bilinear, that is
\begin{align*}
(ax + by) * z &= a (x * z) + b (y * z), \\
%
\quad \text{and} \quad x * (ay + bz) &= a (x * y) + b (x * z)
\end{align*}
for all $x, y, z \in A$ and $a, b \in \F$.
\end{defn}
We tend to focus attention on specific classes of algebra, for example associative algebras being those in which the identity
\[
(x * y) * z = x * (y * z)
\]
holds for all $x, y, z \in A$.


\begin{defn}
A Lie algebra consists of a vector space $\g$ over a field $\F$ and an $\F$-bilinear product $[,] : \g \times \g \rightarrow \g$ called the bracket, which satisfy
\begin{itemize}
\item (Skewsymmetry) $[y, x] = -[x, y]$ for all $x, y \in \g$,

\item (Jacobi identity) $[x, [y, z]] + [y, [z, x]] + [z, [x, y]] = 0$ for all $x, y, z \in \g$.
\end{itemize}
\end{defn}


Homomorphism, subalgebra, ideal, kernel, image.

Representation, subrepresentation, quotient.


\section{Nilpotent and solvable Lie algebras}



\begin{lemma}\label{lem:nilp.has.eigenvector}
Let $V$ be a nonzero vector space and $X \in \en(V)$ nilpotent. Then there exists a nonzero vector $v \in V$ such that $Xv = 0$.
\end{lemma}

\begin{proof}
By hypothesis there exists $n$ such that $X^n = 0$, and we may choose such $n$ to be minimal, i.e., so that $X^{n-1}V \neq 0$. Let $v \in X^{n-1}V$ be a nonzero vector. We have $Xv \in XX^{n-1}V = X^nV = 0$.
\end{proof}


\begin{lemma}
Let $X \in \en(V)$ be nilpotent, then $\ad(X) \in \en(\en(V))$ is nilpotent.
\end{lemma}

\begin{proof}
We consider the endomorphisms $L_X, R_X : \en(V) \rightarrow \en(V)$ defined by
\[
L_X(M) = XM, \quad \text{and} \quad R_X(M) = MX
\]
for $M \in \en(V)$. Notice that $\ad(X) = L_X - R_X$ Since $\en(V)$ is an associative algebra, $L_X$ and $R_X$ commute. We may therefore apply the binomial formula to obtain
\[
\ad(X)^n = \sum_{j=0}^n \binom{n}{j} L_X^j R_X^{n-j},
\]
which is to say
\[
\ad(X)^n M = \sum_{j=0}^n \binom{n}{j} X^j M X^{n-j} \quad \text{for all $M \in \en(V)$}.
\]
Since $X$ is nilpotent it is possible to choose $n$ sufficiently large that in each summand above, either $X^j = 0$ or $X^{n-j} = 0$. Therefore $\ad(X)$ is nilpotent.
\end{proof}






\begin{thm}[Engel's Theorem]
Let $V$ be a nonzero finite dimensional vector space and $\g \subset \gll_V$ a Lie subalgebra. If $X$ is nilpotent for all $X \in \g$ then there exists a nonzero vector $v \in V$ such that $X v = 0$ for all $X \in \g$.
\end{thm}

\begin{proof}
Since $V$ is finite dimensional, so is $\g$. If $\dim(\g) = 0$ then the assertion is true of any nonzero vector $v \in V$. If $\dim(\g) = 1$ then the assertion is Lemma \ref{lem:nilp.has.eigenvector}. We proceed by induction on $\dim(\g)$. Suppose $\dim(\g) = n \geq 2$ and suppose the assertion has been established for all finite dimensional $U$ and $\g \subset \gll_U$ satisfying the hypotheses of the theorem and of dimension at most $n-1$.

Let $\h \subsetneq \g$ be a proper Lie subalgebra of maximal dimension. Such a Lie subalgebra exists and has dimension at least $1$, since $\F x \subset \g$ is a Lie subalgebra for any $x \in \g$. We now consider the restriction to $\h$ of the adjoint representation, i.e.,
\[
\ad|_{\h} : \h \rightarrow \gll_{\g},
\]
turning $\g$ into a representation of $\h$. Since $\h \subset \g$ is a subalgebra, we have $\ad|_{\h}(\h) \subset \h$, i.e., $\h \subset \g$ is a subrepresentation. We consider the quotient representation
\[
\pi : \h \rightarrow \gll_{\g/\h},
\]
and the Lie subalgebra $\im(\pi) \subset \gll_{\g/\h}$.

Since $\g$ consists of nilpotent endomorphisms of $V$, Lemma \ref{} implies that $\ad(X) \in \gll_{\g}$ is nilpotent for each $X \in \g$. It follows that $A \in \gll_{\g/\h}$ is nilpotent for each $A \in \im(\pi)$. Since $\dim(\im(\h)) \leq \dim(\h) < \dim(\g)$, the inductive hypothesis assures us that there exists nonzero $\ov{a} \in \g/\h$ such that $\pi(X)\ov{a} = \ov{0}$ for all $X \in \h$. If $a \in \g$ is a representative of $\ov{a}$ then
\[
[X, a] \in \h \quad \text{for all $X \in \h$},
\]
or rather $[a, \h] \subset \h$.

Now we notice that $\wtil{\h} = \h + \F a \subset \g$ is a Lie subalgebra since $[a, a] = 0$ and $[a, \h] \subset \h$. Since $a \notin \h$ we have $\dim\wtil\h > \dim\h$ and so by maximality of $\dim\h$ we have $\wtil\h = \g$.

We now apply the inductive hypothesis again, this time to assure ourselves of the existence of nonzero $v \in V$ such that $Xv = 0$ for all $X \in \h$. If we can find such $v$ for which $av = 0$ also holds then clearly we will have established the theorem by induction. Thus let us consider the nonzero vector subspace
\[
V_0 = \{v \in V \mid \text{$Xv = 0$ for all $X \in \h$} \} \subset V.
\]
We show that $a(V_0) \subset V_0$. Indeed if $v \in V_0$ and $X \in \h$ then
\[
X a v = a X v + [X, a] v
\]
and since $[X, a] \in \h$ by \eqref{}, the right hand side vanishes. Thus $av \in V_0$.

Finally Lemma \ref{lem:nilp.has.eigenvector} implies that there exists nonzero $v \in V_0$ such that $av = 0$, and we are done.
\end{proof}





\begin{defn}
A \emph{root system} consists of a vector space $V$ over $\R$ of finite dimension $n$, a positive definite symmetric bilinear form $(\cdot, \cdot) : V \times V \rightarrow \R$ and a finite set $\D \subset V \backslash \{0\}$, satisfying the following properties:
\begin{itemize}
\item The $\R$-linear span of $\D$ is $V$,

\item for each pair $\al, \beta \in \D$ the set $\{\beta + n \al \mid n \in \Z\} \cap (\D \cup \{0\})$ takes the form
\[
\{\beta + k \al \mid \text{$k \in \Z$ and $-p \leq k \leq q$}\}
\]
for some pair of non negative integers $p, q$. Furthermore $p$ and $q$ satisfy
\[
p-q = \frac{2(\al, \beta)}{(\al, \al)},
\]

\item for each $\al \in \D$ the equality $\{n \al \mid n \in \Z\} \cap \D = \{\al, -\al\}$ holds.
\end{itemize}
\end{defn}

\begin{defn}
Let $V$ be a finite dimensional vector space over $\R$ and $(\cdot, \cdot) : V \times V \rightarrow \R$ a positive definite symmetric bilinear form. For $\al \in V \backslash \{0\}$ the endomorphism $r_\al \in \en(V)$ defined by
\[
r_\al(\la) = \la - \frac{2(\al, \la)}{(\al, \al)} \al
\]
is called the reflection in $\al$.
\end{defn}

\begin{lemma}
The reflection $r_\al$ is invertible and $r_\al^2 = I_V$.
\end{lemma}

\begin{prop}
Let $\D$ be a root system and $\al \in \D$. Then $r_\al(\D) = \D$.
\end{prop}

\begin{proof}
Since $r_\al$ is invertible it suffices to show that $r_\al(\D) \subset \D$.
\end{proof}



We now describe how to obtain a root system from a semisimple Lie algebra. Let $\g$ be a semisimple Lie algebra of rank $n$, $\h \subset \g$ a Cartan subalgebra, and
\[
\g = \h \oplus \bigoplus_{\al \in \D} \g_\al
\]
the corresponding root space decomposition. Since $\F$ is of characteristic $0$ there is a canonical embedding $\Q \subset \F$ and so we can regard $\h^*$ as a vector space over $\Q$. Let $V_\Q \subset \h^*$ denote the $\Q$-span of $\D$. We claim that the dimension of $V_\Q$ is $n$.

\begin{defn}
Let $\g$ be a semisimple Lie algebra. The root system of $\g$
\end{defn}


\begin{thm}

\end{thm}



\begin{prop}
Let $\D$ be a root system which satisfies the following condition for each pair $\al, \beta \in \D$: There exists a sequence $\gamma_i \in \D$ for $i = 0, 1, \ldots, n+1$ for which $\gamma_0 = \al$, $\gamma_{n+1} = \beta$ and for each $i = 0, 1, \ldots, n$ we have $\gamma_i + \gamma_{i+1} \in \D$. Then $\D$ is indecomposable.
\end{prop}

\begin{proof}
Suppose, on the contrary that $\D$ is decomposable yet satisfies the condition presented in the theorem statement. Then we may take $\al$ and $\beta$ in distinct orthogonal components $\D$, and a sequence of $\gamma_i$. The $\gamma_i$ cannot all be in the same component (call it $\D_0$) as $\gamma_0 = \al$, and so there exists some index $j$ for which $\gamma_j \in \D_0$ and $\gamma_{j+1} \notin \D_0$. But then $\gamma_j + \gamma_{j+1}$ cannot be a root, and this contradiction proves the proposition.
\end{proof}




\end{document}
